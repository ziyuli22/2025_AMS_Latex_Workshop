
\documentclass[12pt]{article}
\pagestyle{plain} \topmargin -0.5in \textheight 9in \textwidth 6.5in
\oddsidemargin -0in
\usepackage{amssymb}
\usepackage{amsthm}
\usepackage{amsmath}
\usepackage{mathrsfs} 
\usepackage{mathtools}
\usepackage{bm}
\usepackage[normalem]{ulem}
\usepackage{enumerate}
\usepackage{mathtools}
\usepackage{graphicx}
\usepackage{hyperref, kantlipsum}
\usepackage{blkarray}
\usepackage{wasysym}
\usepackage{arydshln}
\usepackage[framemethod=TikZ]{mdframed}

% The following are official colors for the Colorado School of Mines

\definecolor{paleBlue}{HTML}{CFDCE9}
\definecolor{lightBlue}{HTML}{879EC3}
\definecolor{blasterBlue}{HTML}{09396C}
\definecolor{darkBlue}{HTML}{21314d}
\definecolor{envrGreen}{HTML}{80C342}
\definecolor{goldenTech}{HTML}{F1B91A}
\definecolor{coloradoRed}{HTML}{CC4628}
\definecolor{lightGray}{HTML}{AEB3B8}
\definecolor{silver}{HTML}{81848A}
\definecolor{darkGray}{HTML}{75757D}
\definecolor{earthBlue}{HTML}{0272DE}
\definecolor{mutedBlue}{HTML}{57A2BD}
\definecolor{energyYellow}{HTML}{F0F600}
\definecolor{redFlannel}{HTML}{B42024}

\input{Definitions.tex}

\newtheorem*{remark}{Remark}
\newtheorem{definition}{Definition}
\newtheorem{theorem}{Theorem}


\begin{document}

\begin{flushleft}

\textbf{Department of Applied Mathematics and Statistics}

\textbf{COLORADO SCHOOL OF MINES}\\


MATH $531$: Theory of Linear Models \\
Your Name
\end{flushleft}

\vspace{0.1in}

\begin{center}
{\bf Homework 06} \\
{\bf Due Friday, February 21, 2025} \\
\end{center}

\vspace{0.1in}
\subsubsection*{Some background: }
Let $\bz$ be distributed $MN(0,I_m)$ then we know that $\bz^T \bz = \sum_{i=1}^m \bz_i^2$ is distributed $\chi^2(m)$.  -- a sum of iid $N(0,1)$ RVs. 
  (We also know that $\chi^2(m)$ is a member of the gamma distribution family with shape $2m$ and scale $2$.)

\begin{enumerate}
\item[$\textbf{Problem 1}$]
Let $M$ be an $n\times n$  projection matrix and let $k$ be the dimension of the subspace that $M$ projects onto. 
 
\begin{enumerate}
	\item For  the eigendecomposition $M= U D U^T$   show that the diagonal elements of $D$ must be either $0$ or $1$.\\ {\it \color{magenta} This is provides an alternative proof that $tr(M) =k$. }
	\item Let  $U$ be an $n \times n$ orthonormal matrix and $\bz$ be distributed $MN(0,I_n)$. Show that $U^T\bz$ is also  distributed $MN(0,I_n)$.
	\item Let $\bz$ be distributed $MN(0,I_n)$  show that  $\bz^T M \bz $ is  distributed $\chi^2(k)$.
\end{enumerate}

\noindent\rule{0.9\textwidth}{0.75pt}

\item[Answer:] 
\begin{enumerate}
	\item We use the fact that... 
	\begin{align*}
		XXXXXXXX &= \text{Something else here}  \\
		XXXXXXXX &= UDU\T
	\end{align*}
	We can see that ...., therefore diagonal elements of $D$ must be either $0$ or $1$. $\qquad \square$
	
	\item $U\T \bz $ satisfies the definition that says $XXXXXX \qquad \square$
	\item \textcolor{coloradoRed}{Comments: \\ When writing proofs, make sure to say Q.E.D. or $\square$ to complete the proof. This symbol that ends the proof should be the last thing on that homework sub-problem. When writing equations, make sure to align all equal signs. Keep every sub-part of the problem on the same page if possible. }

\clearpage

\item[$\textbf{Problem 2}$] What is $2(3 + 7)$? 

\noindent\rule{0.9\textwidth}{0.75pt}

\item[Answer:] 
There are many ways to find this quantity, one way is by distributing first:
	\begin{align*}
		2(3 + 7) &= 2(3) + 2(7) \\
		\shortintertext{And then, multiply}
		&= 6 + 14
		\shortintertext{Then, add the two numbers $6$ and $14$ together we get}
		\Aboxed{2(3 + 7)  &= 20}
	\end{align*}
	\textcolor{coloradoRed}{Comments: \\ It is generally better to start a new problem on a new page. You can make comments while in the aligned equation environment, always box your answers! In aligned envrionments, you have to use \texttt{Aboxed\{\}} instead of \texttt{boxed\{\}} }. 
 
 \end{document}
 
