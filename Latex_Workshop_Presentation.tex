\documentclass[serif,12pt, aspectratio=169]{beamer}
% aspectratio = 169 means 16:9; 43 means 4:3 etc. 

%standard packages and math
\usepackage{pgfpages}
\usepackage{amsmath,amsfonts}
\usepackage{hyperref}
\usepackage{graphicx}
\usepackage{caption}
\usepackage{multimedia}
\usepackage{algorithm} % optional, for algorithms
\usepackage{bbm}

% Mines colors & personalized definitions
\usepackage{xcolor}
\input{Definitions.tex}
% The following are official colors for the Colorado School of Mines

\definecolor{paleBlue}{HTML}{CFDCE9}
\definecolor{lightBlue}{HTML}{879EC3}
\definecolor{blasterBlue}{HTML}{09396C}
\definecolor{darkBlue}{HTML}{21314d}
\definecolor{envrGreen}{HTML}{80C342}
\definecolor{goldenTech}{HTML}{F1B91A}
\definecolor{coloradoRed}{HTML}{CC4628}
\definecolor{lightGray}{HTML}{AEB3B8}
\definecolor{silver}{HTML}{81848A}
\definecolor{darkGray}{HTML}{75757D}
\definecolor{earthBlue}{HTML}{0272DE}
\definecolor{mutedBlue}{HTML}{57A2BD}
\definecolor{energyYellow}{HTML}{F0F600}
\definecolor{redFlannel}{HTML}{B42024}


%bibliography settings
\usepackage[backend=biber, style=numeric, sorting = none]{biblatex}

\addbibresource{Sources.bib}

%------------------------------------------ Title Page ---------------------------------------%
\title[AMS Latex Workshop 2025]{2025 AMS Bootcamp \\ Latex Workshop}
%\subtitle{Poster Speed Talk}
\author[Li] {Ziyu Li\inst{1}}
\institute[Mines]
{\inst{1}Department of Applied Mathematics and Statistics, \protect\\ Colorado School of Mines
}
\vspace{1em}

\date[Bootcamp] {Wednesday \\ August 20th, 2025}



%------------------------------------------- Presentation Setting ------------------------%
\usetheme{Madrid}
\setbeamercolor{section in head/foot}{fg=blasterBlue, bg=white}
\usecolortheme[named=blasterBlue]{structure}
\setbeamertemplate{footline}[frame number]
\makeatletter
\setbeamertemplate{headline}{%
    \begin{beamercolorbox}[ht=2.75ex,dp=3.75ex]{section in head/foot}
        \insertnavigation{\paperwidth}
    \end{beamercolorbox}%
}%
\makeatother

% This adds and outline, if you wish

\AtBeginSection[]
{
  \begin{frame}<beamer>
    \frametitle{Outline}
    \tableofcontents[currentsection]
  \end{frame}
}
%---------------------------------------------------
\begin{document}


\begin{frame}
    \maketitle
\end{frame}

%----------------------------- Introduction Section ----------------------%
\section{Introduction}
\begin{frame}
	\frametitle{What is it?}
	\pause
	Typesetting software originally written in the 1980s. 
	\vspace{0.25cm}
	\pause 
	\begin{columns}[T]
		\begin{column}{0.46\textwidth}
		\textcolor{envrGreen}{Why people love it:}
			\begin{itemize}
				\item[\color{blasterBlue}{$\bullet$}]  Format only once so you can focus on content. 
				\item[\color{blasterBlue}{$\bullet$}]  Minimum and flexible citation management. 
				\item[\color{blasterBlue}{$\bullet$}]  Typing math.
				\item[\color{blasterBlue}{$\bullet$}]  Expected for math journal submissions. 
			\end{itemize}
		\end{column}
		
		\begin{column}{.01\textwidth} 
       		 \rule{.1mm}{0.65\textheight}
   		 \end{column}
		 
		\pause
		\begin{column}{0.46\textwidth}
		\textcolor{coloradoRed}{Why people hesitate to use it:}
		\begin{itemize}
				\item[\color{blasterBlue}{$\diamond$}]  Steep learning curve when getting started, good templates?
				\item[\color{blasterBlue}{$\diamond$}] Overleaf renders too slowly and sometimes full of bugs. 
				\item[\color{blasterBlue}{$\diamond$}]  Different bibliography options and compliers are confusing.
				\item[\color{blasterBlue}{$\diamond$}]  Working with scientists that prefer Word or Google Doc. 
			\end{itemize}
		\end{column}
	\end{columns}
\end{frame}

\begin{frame}
	\frametitle{Editor Software Options}
	\pause
	\textbf{Online Option: \href{http://www.overleaf.com}{Overleaf}}
	\pause
	\begin{itemize}
		\item Collaborate with others in real time. 
		\pause
		\item Shrinks learning curve by automatically suggesting commands. 
		\pause
		\item Sometimes resolve errors automatically or highlight incorrect lines.
		\pause
		\item Great for beginners!
	\end{itemize}
	\pause
	\textbf{Local Option: \href{https://pages.uoregon.edu/koch/texshop/installingtexshop.html}{TeXShop} for Mac, TeXworks, or VSCode}
	\pause
	\begin{itemize}
		\item Much faster than Overleaf and can handle large documents. 
		\pause
		\item More control over your document and easily integrated with code outputs. 
		\pause
		\item Do not need the internet to work. 
		\pause
		\item Great for thesis, projects, books, etc. 
	\end{itemize}
\end{frame}


\begin{frame}
	\frametitle{Agenda}
	\pause
	\begin{enumerate}
		\item \textbf{Setup} \\
		Go to GitHub link: \url{github.com/ziyuli22/2025_AMS_Latex_Workshop} \\
		and download folder. Log into Overleaf account, upload folder. 
		\item \textbf{Common homework commands \& expectations}
		\item \textbf{Poster example}
		\item \textbf{Presentation example}
	\end{enumerate}
\end{frame}

%----------------------------- Homework Template Section ----------------------%

\section{Homeworks}
\begin{frame}
	\frametitle{Template Example}
	\centering
	{\Large Click on \texttt{Homework\_Example.tex}}
\end{frame}

%------------------------------ Poster Template Section ---------------------------%

\section{Posters}
\begin{frame}
	\frametitle{Poster Example}
	\centering
	{\Large Click on \texttt{Poster\_Example.tex}}
\end{frame}


%------------------------------ Presentation Template Section ---------------------------%

\section{Presentations}
\begin{frame}
	\frametitle{Presentations Templates}
	Take a look at  \\ \url{https://deic.uab.cat/~iblanes/beamer_gallery/} \\
	for some default options. 
	\pause
	\\
	This presentation is a modification on the Madrid theme. 
\end{frame}

\begin{frame}
	\frametitle{Add Algorithms \& Use Pause}
\textbf{Data:} $\by$ at $n$ spatial locations $\locs^O = \{\bs^O_1, \bs^O_2, \dots, \bs^O_n\}$.   \\
	\textbf{Goal:} Quantify uncertainty of predictions $\hat{g}$ on evenly spaced grid $\locs^G = \{\bs^G_1, \bs_2^G, \dots, \bs_M^G\}$. \pause
	\begin{algorithm}[H] 
	\caption*{\textbf{Algorithm 1} Conditional Simulation Method}
	{ \small \textbf{Input: } \text{Spatial data $\by$, their locations $\locs^O$, and prediction grid locations $\locs^G$. } }\\
	{ \small \textbf{Output: } \text{Ensemble of $l$ conditional simulations $\bv = \{\bv_1, \dots, \bv_l\}$. $\bv_j \sim MVN(\hat{g}\left(\locs^G), \Sigma_{\hat{g}} \right)$.} } \pause
		\begin{itemize}
		\item { \small Compute spatial prediction at grid $\locs^G$ based on data $\by$, label this $\hat{g}(\locs^G)$.} \pause
		\end{itemize}
		{ \small \noindent \textbf{For $j = 1:l$}}
		\begin{itemize}
		\item { \small Simulate spatial process at the union of locations $\locs^S = \locs^G \cup \locs^O$, label this $g^S(\locs^S)$.}
		\end{itemize}
		{ \small \noindent \textbf{End}}
		\vspace{1pt}
	\end{algorithm}
\end{frame}

\begin{frame}
	\frametitle{Others}
	\begin{itemize}
		\item Everything else is similar to poster. Be mindful how large your picture files are because that can slow down rendering. 
		\item Make sure to cite things too \cite{PlaceHolderSource}.
	\end{itemize}
	
	
	
\end{frame}

%\appendix

\begin{frame}
\frametitle{References}
	\AtNextBibliography{\tiny}
	\printbibliography
\end{frame}

\section*{Extras}
\begin{frame}
	\frametitle{Extra Slides for Questions or other Technical Details}
	Simulation via Cholesky Decompositon \\
	Cholesky decomposition: $\Sigma = BB\T$ \\
	Obtain simulation: $g(\bs) = B\pmb{\epsilon}$, $\pmb{\epsilon} \sim \text{MVN}(0,I)$
	\begin{align*}
	\E[B\pmb{\epsilon}] &= B \E[\pmb{\epsilon}] = 0 \\
	\Var(B\pmb{\epsilon}) &= B \Var(\pmb{\epsilon}) B\T = B I B\T = \Sigma
	\end{align*} 
	
	Sometimes you include more slides than you actually present to be prepare to demonstrate difficult concepts. 
\end{frame}


\end{document}
